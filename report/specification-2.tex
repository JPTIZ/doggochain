\documentclass{article}

\usepackage{cleveref}
\usepackage{relsize}

\title{%
    Doggochain --- A doggo game blockchain \\
    [0.5em]
    \smaller{2nd report}
}
\author{%
    Guilherme Christopher Michaelsen Cardoso (14100831) \\
    João Paulo Taylor Ienczak Zanette (14200743)
}

\begin{document}
    \maketitle

    \section{Contracts structure}

    This section will only describe contracts briefly just to clarify
    \textbf{what} they are (i.e.\ their role in the system). Further
    explanations of \textbf{how} they behave will be cleared
    in~\Cref{sec:function-desc}.

    In short, the contracts directly related to Doggochain are structured as
    follows:

    \begin{description}
        \item[DoggoChain:] The main contract, hence responsible for managing
            database in respect of Players and Doggos, as well as the
            smart-contract's operations (player challenging, hunting, breeding,
            etc.). Abstracts player's data as an internal \texttt{struct},
            since there's no much use for it outside this contract\footnote{%
                The reason is that the \texttt{Player} structure is needed only
                for players to own their Doggos and mark their existence in the
                block-chain (i.e.\ in the game). Every other operation is done
                considering Doggo-to-Doggo interactions.
            }.
        \item[Doggo:] Works as a simple structure for every Doggo's stats,
            providing an interface that abstract how Doggos evolve (as in stat
            gaining) and what's their real current stats.
        \item[UsageTest:] A unit test to check if the system works properly
            from the point of view of who's going to \textit{use} the API\@.
    \end{description}

    Other implementation-specific contracts are designed to compensate some
    lack of features from Solidity 0.5, e.g.\ no proper dynamic array handling
    in storage data. Currently these contracts are:

    \begin{description}
        \item[DoggoList:] A simple list data structure implementation,
            simulated using a mapping from \texttt{uint32} to \texttt{Doggo}
            (since there's no generics in Solidity, this implementation must be
            replicated for every different type) and a \texttt{length} variable
            for element counting.
        \item[OptionalPlayer:] Actually not a contract itself, but a structure.
            Used to compensate the lack of Optional type\footnote{%
                An ``Optional'' is a type that handles existence/inexistence of
                data (without the security/usage issues of using
                \texttt{null}-like value).
            } in Solidity. Needed since Solidity mappings assume \textit{all}
            possible keys actually exist and maps to some value (meaning no
            \texttt{hasKey}-related checking), and it does not belong directly
            to \texttt{Player} structure since existence itself is not a
            player's property, but rather is a mapping's constraint.
    \end{description}

    Apart from that, there are libraries for the more complex operations, so
    the amount of responsibility in DoggoChain contract's code becomes reduced:

    \begin{description}
        \item[Battle:] For battle simulation. Its only public function is
            \texttt{simulate}, which gives a simulated battle's results.
        \item[Breed:] For breeding operations. Its only public function is
            \texttt{breed}, which gives result containing the (possibly)
            generated Doggo and if the breeding was a success.
    \end{description}


    \section{Function details\label{sec:function-desc}}

    This section will be splitted in a subsection for each \textbf{main
    contract and libraries}, since other contracts are merely for
    implementation details.

    \subsection{DoggoChain}


\end{document}
